%%%%%%%%%%%%%%%%%%%%%%%%%%%%%%%%%%%%%%%%%
% Medium Length Professional CV
% LaTeX Template
% Version 2.0 (8/5/13)
%
% This template has been downloaded from:
% http://www.LaTeXTemplates.com
%
% Original author:
% Trey Hunner (http://www.treyhunner.com/)
%
% Important note:
% This template requires the resume.cls file to be in the same directory as the
% .tex file. The resume.cls file provides the resume style used for structuring the
% document.
%
%%%%%%%%%%%%%%%%%%%%%%%%%%%%%%%%%%%%%%%%%

%----------------------------------------------------------------------------------------
%	PACKAGES AND OTHER DOCUMENT CONFIGURATIONS
%----------------------------------------------------------------------------------------

\documentclass{cv_yazdan} % Use the custom resume.cls style

\usepackage[left=0.75in,top=0.6in,right=0.75in,bottom=0.6in]{geometry} % Document margins
\newcommand{\tab}[1]{\hspace{.2667\textwidth}\rlap{#1}}
\newcommand{\itab}[1]{\hspace{0em}\rlap{#1}}
\name{Yazdan Asadi} % Your name
\address{Tehran , Iran \\  github.com/yazdanasadi } 
\address{(+98)-991-939-9712 \\ yazdanasadi19@gmail.com} % Your phone number and email

\begin{document}

%----------------------------------------------------------------------------------------
%	EDUCATION SECTION
%----------------------------------------------------------------------------------------
\begin{rSection}{Objective}
\\ Passionate and motivated researcher with experience and interest in Machine Learning,Deep Learning and Data Science.Enthusiastic about creating a stronger computer science community and I have worked on many related projects
with Recurrent Neural networks and Predictions Using Time Series. 





\end{rSection}

\begin{rSection}{Education}

{\bf National Organization for Development of Exceptional Talents (Sampad)} \hfill {\em 2012 - 2016} 
\\ Highschool Diploma \hfill { Overall GPA: 3.61/4}
\\ Mathematics  

{\bf University of Kurdistan, Sanandaj} \hfill {\em September 2016 - August 2020} 
\\ Undergraduate \hfill { Overall GPA: 3.2/4}
\\ Computer Engineering 

  


\end{rSection}
%----------------------------------------------------------------------------------------
%	TECHNICAL STRENGTHS SECTION
%----------------------------------------------------------------------------------------

\begin{rSection}{Technical Strengths}

\begin{tabular}{ @{} >{\bfseries}l @{\hspace{6ex}} l }
Computer Languages &  C/C++, Python, MATLAB, R \\
Technical Tools &   Tensorflow, Keras, Pytorch, Scikit-Learn, Numpy, Pandas \\
Database & MySQL,SQL Server \\
Operating System & Linux(fluent in Ubuntu) \\
Other Skills & LaTeX, Microsoft Office, CUDA, Apache Spark \\
Language Skills & English(Fluent) \\
       & Persian(Native) \\
       & Kurdish(Native)
\end{tabular}

\end{rSection}

%----------------------------------------------------------------------------------------
%	WORK EXPERIENCE SECTION
%----------------------------------------------------------------------------------------

\begin{rSection}{Experience}

\begin{rSubsection}{University of Kurdistan}{March 2020 - September 2020}{Undergraduate Research Project}{}
\item Predicting the probability of students scores based on the last two years of their academic activity and previous grades of them
\item Implementing the Classification and Regression algorithms and compare their performance on Data-sets
\item Analysed and implemented the Recurrent Neural Network layer for predicting the next sequence of the scoring system
\end{rSubsection}


%------------------------------------------------

\begin{rSubsection}{University of Kurdistan}{November 2019 - February 2020}{Music Composition based on Heartbeat and Emotional Data-IoT Project}{}
\item Implementing the gadget for sending the user heartbeat to the computer with IoT modules  
\item Designed a Deep-learning Model to predict the music sequence based on the heartbeat vector 
\item Received the top score for IoT course among 45 students
\end{rSubsection}

\begin{rSubsection}{University of Kurdistan}{September 2019 - January 2020}{Teaching Assistant of Artificial Intelligence Course}{}
\item Explaining the theories of Artificial Intelligence behind different practical exercises to 3rd-year Undergraduate students

\end{rSubsection}

\end{rSection}



%----------------------------------------------------------------------------------------
\begin{rSection}{Relevant Courses}
\itab{\textbf{Core Courses}} \tab{}  \tab{\textbf{Grade}}
\\ \itab{Information Retrieval} \tab{}  \tab{19/20}
\\ \itab{Data Mining} \tab{}  \tab{18.5/20} 
\\ \itab{Data Science} \tab{}  \tab{19/20} 
\\ \itab{Artificial Intelligence} \tab{} \tab{16/20}
\\ \itab{Internet of Things} \tab{} \tab{19.5/20}
\\ \itab{Probability and Statistics} \tab{} \tab{18/20}
% \\ \itab{Process Control (ongoing)} \tab{} \tab{Electrodynamics}

\end{rSection}

\begin{rSection}{Licenses & Certifications}

\begin{rSubsection}{Big Data Analysis with Scala and Spark by École Polytechnique Fédérale de Lausanne on Coursera }{ March 2020}{www.coursera.org/account/accomplishments/certificate/BWEZ48KJLE7Q}

\end{rSubsection}

%------------------------------------------------

\begin{rSubsection}{Sequence Models by deeplearning.ai on Coursera}{December 2019}{www.coursera.org/account/accomplishments/certificate/53PUAM8XLXM7}

\end{rSubsection}

%------------------------------------------------

\begin{rSubsection}{Introduction to TensorFlow for Artificial Intelligence, Machine Learning, and Deep Learning }{December 2019}{www.coursera.org/account/accomplishments/certificate/W4L4CK92KQWF}

\end{rSubsection}


\begin{rSubsection}{Neural Networks and Deep Learning by deeplearning.ai on Coursera }{November 2019}{www.coursera.org/account/accomplishments/certificate/7835JA28TVZ9}

\end{rSubsection}





\begin{rSubsection}{Machine Learning by Stanford University on Coursera}{September 2019}{www.coursera.org/account/accomplishments/certificate/NMWGCR2MCHZ7}

\end{rSubsection}

\begin{rSubsection}{R for Data Science}{August 2019}{University of Tehran}

\end{rSubsection}

\end{rSection}

%----------------------------------------------------------------------------------------


\end{document}

